\chapter{Analysis}\label{chap:analysis}
\section{Reasons not to use \LaTeX}
{\LaTeX} is a typesetting tool. It's purpose is to provide pretty formatted documents. As such, it is a very appropriate tool for dissertation writing. Dissertations are single-authored documents, and while the Graduate Division may review the document to ensure that it meets university formatting guidelines, the author is typically responsible for achieving that formatting. The onus is on the author to take care of the creation of a table of contents, formatting  and numbering of linguistic examples,  internal cross-referencing, management bibliographic references---and many other tasks which would typically be handled by a publisher. With no publisher (and editor!) to intervene, {\LaTeX} fills the gap. For shorter documents, similar results can be achieved using word processors such as Microsoft Word. However, the use of word processors does not scale well for typesetting large documents such as dissertations.

\subsection{{\LaTeX} for journal publications}
Many linguistic publishers still make use of proprietary typesetting tools and workflows which require authors to submit final copies of manuscripts as a word processing document (often MS Word or RTF). While these publishers may permit authors to submit manuscripts for review as PDF's, once the review process is complete and final edits have been approved, they will request a word processing version of the document. Typically, this version will be required to have minimal formatting, e.g., no automatic numbering, interlinear examples tab-separated or in tables, etc.  This version will then be used by the publisher to create the final layout. The problem for {\LaTeX} users is that there is no reliable method to export from {\LaTeX} to a word processing format such as MS Word. It is possible to use Adobe Acrobat to export the \LaTeX-generated pdf to .docx, but extensive cleanup of the exported document may be required. 

Increasingly, many publishers are using {\LaTeX} for their in-house typesetting, and these publishers may accept and even encourage submissions in {\LaTeX} (with a pdf for review). Moreover, the move toward open-access publishing has resulted in publishers placing more of the type-setting burden on authors. For example, Language Sciences Press strongly encourages {\LaTeX} submissions. Bottom line: check with the publisher before you get too far along in the writing process. If the publisher requires submission in a word processing format, then you may find it easier to use a word processor to write your article. Use the right tool for the job.

\subsection{Co-authoring and collaboration with \LaTeX}
Collaboration is the norm these days, both within Linguistics and across the sciences more broadly.
While the dissertation is a single-authored document, it is likely that many if not most of your publications will be co-authored. 
The emergence of online {\LaTeX} compilers such as \href{http:overleaf.com}{Overleaf} has made it much easier to collaborate---even incorporating commenting and tracked changes. 
However, co-authoring with {\LaTeX} generally requires all authors to be proficient in the use of {\LaTeX}.  If one or more authors are new to {\LaTeX}, then depending on the length of the collaboration, bringing them up to speed may be impractical. The team may find it easier to adopt a common word processing platform instead.

One occasionally encounters the reverse scenario, where one or more authors profess an inability to use a word processor and insist on using \LaTeX. I find that such claims typically a philosophical stance rather than statements about ability. In general, it is easier for a {\LaTeX} user to learn to use a word processor than the other way around. However, there may be ulterior motives, such as a word processor user expressing a desire to learn \LaTeX. In this case the co-authoring process may represent an excellent learning opportunity.  Bottom line: adopt a writing tool which will work for all members of the team. In many cases this tool will not be \LaTeX. Co-authorship requires compromise. 


\section{Citations and bibliography management}\label{sec:citations}
\LaTeX\ interfaces with very powerful bibliographic management tools. See the file references.bib for an idea of how citations are stored. There are many tools available for managing citations, such as \href{https://bibdesk.sourceforge.io/}{BibDesk}. You can also export from common tools such as Zotero.  There are many examples of the use of citations within this document.

Next to the formatting of linguistic examples, the ability to efficiently manage references is probably the most powerful reason to use {\LaTeX} for writing your dissertation. 