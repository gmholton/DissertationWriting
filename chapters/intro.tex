\chapter{Introduction}\label{chap:intro}
This document is intended as a simple introduction to writing linguistics dissertations using {\LaTeX}. Individual chapters are stored as separate files in the chapters folder and then included in the main.tex document.  This is just a simple quick-start guide. For good tutorial see the guide on 
\href{https://www.overleaf.com/learn/latex/How_to_Write_a_Thesis_in_LaTeX_(Part_1):_Basic_Structure}{Overleaf.com}.

{\LaTeX} is not a WYSIWYG (what you see is what you get) word processor. Instead you enter instructions (``source code'') which tell the {\LaTeX} how to compile your document. In most cases, {\LaTeX} does exactly what you tell it to do, though it can sometimes be difficult to make it do what you want it to do. 

To make the most of this document, copy the project into Overleaf (or any other {\LaTeX} compiler) and compare the raw code with the compiled result. Play around with the code and see how your changes compile. 

\section{First Section}\label{sec:fist_section_label}


{\LaTeX} ignores extra white space, so don't worry about spaces. 

However, you can use \verb|\hfill| \hfill to force text to the right side of the page. 

Separate paragraphs with at least two carriage returns. If you want extra space following a paragraph, end it with double backslash \verb|\\|. \\

\noindent
If you don't want a paragraph to indent, place the \verb|\noindent| command before it.\\

Be careful with special characters such as  \&, \%, \$, etc. These have special functions in {\LaTeX}, so to use them as regular characters you need to escape them with a backslash, i.e., \verb|\&|, \verb|\%|, \verb|\$|, etc.

The percent sign is used for comments. This can be particularly useful as you are writing, as instead of deleting text you can just ``comment it out.''

% Not sure I want to include this list ..
% \begin{itemize}
%     \item 
%     \item 
% \end{itemize}

Note that by default {\LaTeX} will change plain apostrophes ( \textquotesingle\ and \textquotedbl\ ) into ``smart'' or ``curly'' quotes ( ' and ''\ ). To get left side quotes ( ` and ``\ ) you'll need to use the backtick ( \textasciigrave\ ).


\subsection{My subsection}\label{sec:first}
The \verb|itemize| environment is useful for creating bulleted lists:
\begin{itemize}
    \item Use \verb|\subsection{}| to mark second level headings
    \item Use \verb|\subsubsection{}| to mark third level headings
    \item Use \verb|\paragraph{}| to mark fourth level headings 
\end{itemize}

\noindent
Label all of your sections, subsection, paragraphs, etc. with a mnemonic label useing \verb|\label{}| following the section. This facilitates cross-references to other parts of the document, such as the discussion of figures in §\ref{sec:figures}.

\subsection{My second subsection}\label{sec:second}
Never have just one subsection.


\section{Second Section}
Label all your sections and subsections so that you can make use of cross-referencing. For example, formatting of interlinear examples is discussed in §\ref{sec:interlinear}.